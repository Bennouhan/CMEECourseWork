\section{Neural Networks for Population Genetics: Demographic Inference and Data Generation}

\textbf{Speaker:} Flora Jay (\textit{French National Centre for Scientific Research})\\
\textbf{Date:} \hspace{.53cm} 11\textsuperscript{th} Feburary 2021
\vspace{.5cm}


So:
fill in title, speaker, date
summary of the work (in bold?)
background info from wiki about any of it?

and then:

state answer the questions they asked if they did
or give a summary of each section, and give each section apropriate titles

then any future work if applicable




low dimensionality of summary statistics leads to information loss; too high leads to curse of dimensionality, and is computationally expensive
line between ABC and machine learning is not clear

can use DNN on summary statistics from dataset to generate parameters (semi-autonomous simmary statistics), which are themselves a very good input for applicable

DL on raw genetic data - often with convnets bypassess hand-made features
                       - not using summary statistics, using it straight on the genetic data
                       - see 37:00 for diagrams, showing convnets on DNA sequences
                       - uses motif scans gathered by CNN for indiv genomes
                       - or for population genetics, can use CNN to detect recombination hotspots for the population 

her goals: inferring demographic history - global (whole genome) rather than localised regions of genome, for the population
           also I think comparing efficacy of using NNs on summary stats vs raw data
           used simulated data from msprime
           43:00 for diagram of their work, making predictions using various methods
           45:30 for their NN architecture, using snps and distances between them...
           how to be invariant (i think == avoid overfitting)? 3 solutions, one of which is data augmentation 48:00

alex's question at about 1:16:30
